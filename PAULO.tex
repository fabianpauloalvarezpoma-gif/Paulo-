\documentclass[12pt,a4paper]{article}
\usepackage[spanish]{babel}
\usepackage[utf8]{inputenc}
\usepackage[T1]{fontenc}
\usepackage{geometry}
\usepackage{listings}
\usepackage{xcolor}
\usepackage{longtable}
\usepackage{hyperref}

\geometry{margin=2.5cm}

\definecolor{codegray}{rgb}{0.95,0.95,0.95}

\lstset{
    backgroundcolor=\color{codegray},
    basicstyle=\ttfamily\small,
    frame=single,
    breaklines=true
}

\title{Manual De Comandos}
\author{Estudiante: Alvarez Poma Fabian Paulo}


\begin{document}

\maketitle
\tableofcontents
\newpage

%------------------------------------------------
\section{Comandos Básicos (Navegación y Archivos)}

\subsection{Gestión de Directorios}

\begin{longtable}{|p{3cm}|p{10cm}|}
\hline
\textbf{Comando} & \textbf{Descripción} \\
\hline
ls & Lista el contenido de un directorio. \\
cd & Cambia de directorio. \\
mkdir & Crea un nuevo directorio. \\
pwd & Muestra la ruta actual. \\
\hline
\end{longtable}

\subsubsection*{Ejemplos}

\textbf{ls}
\begin{lstlisting}
ls
ls -l
ls -a
\end{lstlisting}

\textbf{cd}
\begin{lstlisting}
cd /home
cd ..
cd Documentos
\end{lstlisting}

\textbf{mkdir}
\begin{lstlisting}
mkdir proyectos
mkdir -p cursos/linux
\end{lstlisting}

\textbf{pwd}
\begin{lstlisting}
pwd
\end{lstlisting}

%------------------------------------------------
\subsection{Manipulación de Archivos}

\begin{longtable}{|p{3cm}|p{10cm}|}
\hline
\textbf{Comando} & \textbf{Descripción} \\
\hline
cp & Copia archivos o directorios. \\
mv & Mueve o renombra archivos. \\
rm & Elimina archivos o directorios. \\
touch & Crea un archivo vacío. \\
cat & Muestra el contenido de un archivo. \\
\hline
\end{longtable}

\subsubsection*{Ejemplos}

\textbf{cp}
\begin{lstlisting}
cp archivo.txt copia.txt
cp -r carpeta respaldo/
\end{lstlisting}

\textbf{mv}
\begin{lstlisting}
mv archivo.txt nuevo.txt
mv archivo.txt /tmp/
\end{lstlisting}

\textbf{rm}
\begin{lstlisting}
rm archivo.txt
rm -r carpeta
\end{lstlisting}

\textbf{touch}
\begin{lstlisting}
touch nuevo.txt
\end{lstlisting}

\textbf{cat}
\begin{lstlisting}
cat archivo.txt
cat archivo1.txt archivo2.txt
\end{lstlisting}

%------------------------------------------------
\subsection{Ayuda y Manuales}

\begin{longtable}{|p{3cm}|p{10cm}|}
\hline
\textbf{Comando} & \textbf{Descripción} \\
\hline
man & Muestra el manual de un comando. \\
help & Ayuda interna del shell. \\
\hline
\end{longtable}

\begin{lstlisting}
man ls
help cd
\end{lstlisting}

%------------------------------------------------
\section{Comandos Avanzados (Sistema y Filtros)}

\begin{longtable}{|p{3cm}|p{10cm}|}
\hline
\textbf{Comando} & \textbf{Descripción} \\
\hline
chmod & Cambia permisos de archivos o directorios. \\
chown & Cambia el propietario o grupo de un archivo. \\
grep & Busca patrones de texto dentro de archivos. \\
find & Busca archivos o directorios en el sistema. \\
head & Muestra las primeras líneas de un archivo. \\
tail & Muestra las últimas líneas de un archivo. \\
sort & Ordena líneas de texto. \\
wc & Cuenta líneas, palabras y caracteres. \\
top & Muestra procesos en tiempo real. \\
ps & Lista procesos activos. \\
kill & Finaliza procesos por ID. \\
\hline
\end{longtable}


\subsection{Ejemplos}

\textbf{chmod}
\begin{lstlisting}
chmod 755 script.sh
chmod +x programa.sh
chmod 644 archivo.txt
\end{lstlisting}

\textbf{chown}
\begin{lstlisting}
chown usuario archivo.txt
chown usuario:grupo archivo.txt
sudo chown -R usuario carpeta/
\end{lstlisting}

\textbf{grep}
\begin{lstlisting}
grep "error" log.txt
grep -i "linux" archivo.txt
grep -r "config" /etc/
\end{lstlisting}

\textbf{find}
\begin{lstlisting}
find /home -name "*.txt"
find . -type d
find / -size +100M
\end{lstlisting}

\textbf{head}
\begin{lstlisting}
head archivo.txt
head -n 5 archivo.txt
head -c 20 archivo.txt
\end{lstlisting}

\textbf{tail}
\begin{lstlisting}
tail archivo.txt
tail -n 10 archivo.txt
tail -f log.txt
\end{lstlisting}

\textbf{sort}
\begin{lstlisting}
sort nombres.txt
sort -r numeros.txt
sort -n numeros.txt
\end{lstlisting}

\textbf{wc}
\begin{lstlisting}
wc archivo.txt
wc -l archivo.txt
wc -w archivo.txt
\end{lstlisting}

\textbf{top}
\begin{lstlisting}
top
top -u usuario
top -n 1
\end{lstlisting}

\textbf{ps}
\begin{lstlisting}
ps
ps aux
ps -ef
\end{lstlisting}

\textbf{kill}
\begin{lstlisting}
kill 1234
kill -9 1234
killall firefox
\end{lstlisting}


%------------------------------------------------
\section{Programación Shell (Bash Scripting)}

\subsection{Estructura Básica}

\begin{lstlisting}
#!/bin/bash
echo "Hola Mundo"
\end{lstlisting}

\subsection{Variables y Argumentos}

\begin{lstlisting}
#!/bin/bash
nombre=$1
echo "Hola $nombre"
\end{lstlisting}

\subsection{Estructuras de Control}

\textbf{If-Else}
\begin{lstlisting}
if [ -f archivo.txt ]; then
  echo "Existe"
else
  echo "No existe"
fi
\end{lstlisting}

\textbf{For}
\begin{lstlisting}
for archivo in *.txt; do
  echo $archivo
done
\end{lstlisting}

\textbf{While}

\begin{lstlisting}
contador=1
while [ $contador -le 5 ]; do
  echo $contador
  contador=$((contador+1))
done
\end{lstlisting}


\subsection{Script Funcional: Backup Automatizado}

\begin{lstlisting}
#!/bin/bash

# Variables
origen="/home/usuario/documentos"
destino="/home/usuario/respaldo"
fecha=$(date +%Y-%m-%d)

# Crear carpeta destino con fecha
mkdir -p "$destino/$fecha"

# Copiar archivos
cp -r "$origen/"* "$destino/$fecha/"

# Verificar si el backup fue exitoso
if [ $? -eq 0 ]; then
  echo "Backup completado correctamente en $destino/$fecha"
else
  echo "Error al realizar el backup"
fi
\end{lstlisting}



%------------------------------------------------
\section{Guía de Diagnóstico (Errores Comunes)}

En Linux, los errores son mensajes generados por el sistema cuando una operación no puede ejecutarse correctamente. A continuación se presentan errores comunes, cómo identificarlos y cómo solucionarlos.

\subsection*{1. Permission denied}

\textbf{Mensaje típico:}
\begin{lstlisting}
bash: ./script.sh: Permission denied
\end{lstlisting}

\textbf{Causa:}
El usuario no tiene permisos de ejecución o acceso al archivo.

\textbf{Solución:}
\begin{lstlisting}
chmod +x script.sh
sudo ./script.sh
\end{lstlisting}

\textbf{Explicación:}
Linux controla el acceso mediante permisos (lectura, escritura y ejecución). Si no se tiene permiso, el sistema bloquea la acción.

%------------------------------------------------

\subsection*{2. Command not found}

\textbf{Mensaje típico:}
\begin{lstlisting}
comando_incorrecto: command not found
\end{lstlisting}

\textbf{Causa:}
El comando está mal escrito o no está instalado.

\textbf{Solución:}
\begin{lstlisting}
which nombre_comando
sudo apt install nombre_paquete
\end{lstlisting}

\textbf{Explicación:}
El sistema busca el comando en las rutas definidas en la variable \$PATH.

%------------------------------------------------

\subsection*{3. No such file or directory}

\textbf{Mensaje típico:}
\begin{lstlisting}
ls: cannot access 'archivo.txt': No such file or directory
\end{lstlisting}

\textbf{Causa:}
La ruta es incorrecta o el archivo no existe.

\textbf{Solución:}
\begin{lstlisting}
ls
pwd
ls /ruta/completa/archivo.txt
\end{lstlisting}

\textbf{Explicación:}
Linux distingue entre rutas relativas y absolutas. Si la ubicación es incorrecta, el sistema no encuentra el archivo.

%------------------------------------------------

\subsection*{4. Directory not empty}

\textbf{Mensaje típico:}
\begin{lstlisting}
rmdir: failed to remove 'carpeta': Directory not empty
\end{lstlisting}

\textbf{Causa:}
Se intenta eliminar un directorio que contiene archivos.

\textbf{Solución:}
\begin{lstlisting}
rm -r carpeta
\end{lstlisting}

\textbf{Explicación:}
El comando rmdir solo elimina directorios vacíos.

%------------------------------------------------

\subsection*{5. Syntax error}

\textbf{Mensaje típico:}
\begin{lstlisting}
./script.sh: line 5: syntax error near unexpected token
\end{lstlisting}

\textbf{Causa:}
Error en la estructura del script (espacios incorrectos, falta de fi, done, etc.).

\textbf{Solución:}
\begin{lstlisting}
bash -n script.sh
\end{lstlisting}

\textbf{Explicación:}
El comando bash -n permite verificar errores de sintaxis sin ejecutar el script.

%------------------------------------------------
\section{Tuberías y Redireccionamientos}

\subsection{Conceptos}

\begin{itemize}
\item stdin: Entrada estándar.
\item stdout: Salida estándar.
\item stderr: Salida de errores.
\end{itemize}

\subsection{Operadores}

\begin{itemize}
\item | : Envía la salida de un comando como entrada de otro (pipe).
\item > : Redirige la salida estándar y sobrescribe el archivo.
\item >> : Redirige la salida estándar y añade al final del archivo.
\item 2> : Redirige la salida de errores (stderr) a un archivo.
\end{itemize}


\subsection{Ejemplos Obligatorios}

\textbf{1. Filtro combinado}
\begin{lstlisting}
ls | grep archivo.txt
\end{lstlisting}

\textbf{2. Conteo y ordenamiento}
\begin{lstlisting}
cat texto.txt | sort | uniq | wc -l
\end{lstlisting}

\textbf{3. Registro de errores}
\begin{lstlisting}
ls /directorio_inexistente 2> errores.log
\end{lstlisting}

%------------------------------------------------
\section*{Declaración de uso de inteligencia artificial}

El presente trabajo fue desarrollado con el apoyo de la herramienta de inteligencia artificial \textbf{ChatGPT}, la cual fue utilizada como recurso complementario para la orientación en la redacción y estructuración del contenido. El autor es responsable de la selección, verificación y adecuación final de la información presentada, garantizando su originalidad y validez.

\end{document}